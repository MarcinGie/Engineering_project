\chapter{Wprowadzenie}
\label{cha:wprowadzenie}

\textbf{Jak sama nazwa wskazuje „Wstep” jest ta czescia pracy (raportu), która pisze sie na koncu -
wtedy, kiedy jest juz wiadomo, jakie tresci zostanc przedstawione w pracy. Oczywiscie nie
oznacza to, ze maja tu byc przedstawione jakiekolwiek wyniki! Istotne natomiast jest
pokazanie we wstepie logicznej konstrukcji pracy, czyli „co z czego bedzie wynikac”.}

Bezpieczeństwo linii kolejowych jest obecnie osiągane gównie poprzez zautomatyzowane systemy wykorzystujące specjalistyczną infrastrukturę. Pomimo tego wiele zadań wymaga nadal decyzji i działań podjętych przez ludzi. W celu ciągłej poprawy bezpieczeństwa, zwłaszcza podczas ewentualnej nieuwagi załogi rozwijane są systemy automatycznej detekcji znaków i ostrzegania. Stało się tematem wielu projektów badawczych. Pojawiły się w ostatnich latach na rynku mobilne systemy skanujące, które coraz częściej używane są do zdobywania danych dotyczących obiektów liniowych, takich jak drogi i tory kolejowe. Dane te w formie chmur punktów oraz obrazów cyfrowych składają się na wielkie zbiory danych. Wymaga to aplikacji automatycznych metod przetwarzania danych, zawierających między innymi detekcję znaków kolejowych, które można znaleźć na drodze podczas wykonywanych pomiarów.

%---------------------------------------------------------------------------

\section{Cele pracy}
\label{sec:celePracy}

Celem poniższej pracy jest wybranie na podstawie badań literaturowych i sprawdzenie kilku metod segmentacji a następnie porównanie wyników z metodą referencyjną, wykorzystywaną w Laboratorium Biocybernetyki. Testy będą odbywać się na wstępnie przygotowanych danych, wykorzystywanych w metodzie referencyjnej.

%\subsection{Jakiś tytuł}

%\subsubsection{Jakiś tytuł w subsubsection}

%\subsection{Jakiś tytuł 2}

%---------------------------------------------------------------------------

\section{Zawartość pracy}
\label{sec:zawartoscPracy}

Rozdział 2 Rys historyczny i podstawy teoretyczne - zaczyna się rysem historycznym rozpoznawania znaków na obrazach. Następnie dokonano krótkiego wprowadzenia w teorię przetwarzania obrazów. Przedstawiono podstawowe modele reprezentacji kolorów oraz operacje wykonywane w celu przetwarzania obrazów.
Rozdział 3 Badania literaturowe - rozdział przybliża istniejące rozwiązania detekcji znaków na obrazach, a w szczególności znaków kolejowych.
Rozdział 4 Koncepcja proponowanego rozwiązania - opisuje metodę referencyjną detekcji znaków kolejowych klasy W11p, wykorzystywaną w Laboratorium Biocybernetyki AGH oraz wybrane do porównania algorytmy segmentacji(!!) obrazów przy użyciu koloru połączone z wykryciem znaków. WHAT?
Rozdział 5 Realizacja i wyniki testowania - 
Rozdział 6 Podsumowanie i perspektywy- rozdział jest podsumowaniem pracy. Zaprezentowano w nim wyniki, porównania i wnioski.
W dodatku A coś tam coś tam.
