\chapter{Koncepcja proponowanego rozwiązania}

Ta czesc pracy przeznaczona jest na prezentacja ogólnej koncepcji rozwiazania problemu,
wraz z jej uzasadnieniem (dlaczego wybrano takie a nie inne metody). Jezli zachodzi taka
potrzeba rozdział nalezy rozbic na podrozdziały - tak, aby opisac po kolei poszczególne fazy
algorytmu. Algorytm nalezy opisac i uzasadnic słownie (ewentualnie wspierajac sie wzorami) a
nastepnie zilustrowac w formie przedstawienia kolejnych czynnosci do realizacji, zapisu w
pseudo kodzie lub w formie schematu blokowego.

Brak sensownej zawartości ;)

Jako rozwiązanie zaimplementowano dwa algorytmy których celem będzie wyodrębnienie poszukiwanych znaków kolejowych W11p z zadanych obrazów. Wyniki tych algorytów zostaną przemielone przez dopasowanie do wzorca. Obrazy testowe oraz maski zostały dostarczone przez promotora. 

Jako pierwszy zaimplenemtowany został alogrytm progowania obrazu wykorzystujący przestrzeń HSV oraz progi dla każdego z kanałów otrzymane metodą prób i błędów.

Drugim z algorytmów jest algorytm centroidów (k-srednich). Tu powinien znaleźć się opis algorytmu.s