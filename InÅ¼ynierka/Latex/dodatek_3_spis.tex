\chapter{Spis zawartoci dolaczonych nosnikow}

W poszczególnych katalogach nosnika nalezy umiescic:
·  w zaleznosci od rodzaju projektu - przykładowe lub wszystkie dane (obrazy, filmy), bedace podstawa tworzenia i testowania algorytmu (z ew. odwołaniem sie do materiału analogowego - tytułu nagranej kasety video),
·  SRC - postacie zródłowe stworzonych procedur wraz z projektem, makefile’m itp.
·  EXE - postac programu gotowa do uruchomienia wraz z ew. plikami konfiguracyjnymi lub
innymi niezbednymi komponentami,
·  DOC - tekst raportu w postaci elektronicznej edytowalnej oraz jako PDF (pdf jest
wymagany przez dziekanat).
·  LITERATURA – materiały w postaci elektronicznej (artykuły, inne projekty) które zostały
zgromadzone w zwiazku z realizowanym projektem (praca dyplomowa).