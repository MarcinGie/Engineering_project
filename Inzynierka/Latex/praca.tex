\documentclass[11pt]{aghdpl}
% \documentclass[en,11pt]{aghdpl}  % praca w języku angielskim
\usepackage[polish]{babel}
% \usepackage[english]{babel}
\usepackage[utf8]{inputenc}

%---------------------------------------------------------------------------
% dodatkowe pakiety

\usepackage{enumerate}
\usepackage{listings}
\lstloadlanguages{TeX}
\usepackage{textcomp}
\usepackage{listings}
\usepackage{color}
\usepackage{url}

\lstset{
  literate={ą}{{\k{a}}}1
           {ć}{{\'c}}1
           {ę}{{\k{e}}}1
           {ó}{{\'o}}1
           {ń}{{\'n}}1
           {ł}{{\l{}}}1
           {ś}{{\'s}}1
           {ź}{{\'z}}1
           {ż}{{\.z}}1
           {Ą}{{\k{A}}}1
           {Ć}{{\'C}}1
           {Ę}{{\k{E}}}1
           {Ó}{{\'O}}1
           {Ń}{{\'N}}1
           {Ł}{{\L{}}}1
           {Ś}{{\'S}}1
           {Ź}{{\'Z}}1
           {Ż}{{\.Z}}1
}

%---------------------------------------------------------------------------

\author{Marcin Grześkowiak}
\shortauthor{M. Grześkowiak}

\titlePL{Detekcja znak�w kolejowych klasy W11p na podstawie koloru}
\titleEN{Detection of the W11p railway signs based on color}

\shorttitlePL{Detekcja znak��w kolejowych klasy W11p na podstawie koloru} % skr�cona wersja tytu?�u je?li jest bardzo długi
\shorttitleEN{Thesis in \LaTeX}

\thesistype{Praca dyplomowa in?ynierska}
%\thesistype{Master of Science Thesis}

\supervisor{dr in?. Zbigniew Mikrut}
%\supervisor{Zbigniew Mikrut PhD, DSc}

\degreeprogramme{Informatyka}
%\degreeprogramme{Computer Science}

\date{2015}

\department{Katedra Automatyki i In?ynierii Biomedycznej}
%\department{Department of Automatics and Bioengineering}

\faculty{Wydzia? Elektrotechniki, Automatyki,\protect\\[-1mm] Informatyki i In?ynierii Biomedycznej}
%\faculty{Faculty of Electrical Engineering, Automatics, Computer Science and Biomedical Engineering}

\acknowledgements{Sk?adam serdeczne podzi?kowania Promotorowi dr inż. Zbigniewowi Mikrutowi za cierpliwo??� i po?�wi?cony czas.}


\setlength{\cftsecnumwidth}{10mm}

%---------------------------------------------------------------------------
\setcounter{secnumdepth}{4}

\begin{document}

\titlepages

\setcounter{tocdepth}{3}
\tableofcontents
\clearpage

\chapter{Wprowadzenie}
\label{cha:wprowadzenie}

\textbf{Jak sama nazwa wskazuje „Wstep” jest ta czescia pracy (raportu), która pisze sie na koncu -
wtedy, kiedy jest juz wiadomo, jakie tresci zostanc przedstawione w pracy. Oczywiscie nie
oznacza to, ze maja tu byc przedstawione jakiekolwiek wyniki! Istotne natomiast jest
pokazanie we wstepie logicznej konstrukcji pracy, czyli „co z czego bedzie wynikac”.}

Bezpieczeństwo linii kolejowych jest obecnie osiągane gównie poprzez zautomatyzowane systemy wykorzystujące specjalistyczną infrastrukturę. Pomimo tego wiele zadań wymaga nadal decyzji i działań podjętych przez ludzi. W celu ciągłej poprawy bezpieczeństwa, zwłaszcza podczas ewentualnej nieuwagi załogi rozwijane są systemy automatycznej detekcji znaków i ostrzegania. Stało się tematem wielu projektów badawczych. Pojawiły się w ostatnich latach na rynku mobilne systemy skanujące, które coraz częściej używane są do zdobywania danych dotyczących obiektów liniowych, takich jak drogi i tory kolejowe. Dane te w formie chmur punktów oraz obrazów cyfrowych składają się na wielkie zbiory danych. Wymaga to aplikacji automatycznych metod przetwarzania danych, zawierających między innymi detekcję znaków kolejowych, które można znaleźć na drodze podczas wykonywanych pomiarów.

%---------------------------------------------------------------------------

\section{Cele pracy}
\label{sec:celePracy}

Celem poniższej pracy jest wybranie na podstawie badań literaturowych i sprawdzenie kilku metod segmentacji a następnie porównanie wyników z metodą referencyjną, wykorzystywaną w Laboratorium Biocybernetyki. Testy będą odbywać się na wstępnie przygotowanych danych, wykorzystywanych w metodzie referencyjnej.

%\subsection{Jakiś tytuł}

%\subsubsection{Jakiś tytuł w subsubsection}

%\subsection{Jakiś tytuł 2}

%---------------------------------------------------------------------------

\section{Zawartość pracy}
\label{sec:zawartoscPracy}

Rozdział 2 Rys historyczny i podstawy teoretyczne - zaczyna się rysem historycznym rozpoznawania znaków na obrazach. Następnie dokonano krótkiego wprowadzenia w teorię przetwarzania obrazów. Przedstawiono podstawowe modele reprezentacji kolorów oraz operacje wykonywane w celu przetwarzania obrazów.
Rozdział 3 Badania literaturowe - rozdział przybliża istniejące rozwiązania detekcji znaków na obrazach, a w szczególności znaków kolejowych.
Rozdział 4 Koncepcja proponowanego rozwiązania - opisuje metodę referencyjną detekcji znaków kolejowych klasy W11p, wykorzystywaną w Laboratorium Biocybernetyki AGH oraz wybrane do porównania algorytmy segmentacji(!!) obrazów przy użyciu koloru połączone z wykryciem znaków. WHAT?
Rozdział 5 Realizacja i wyniki testowania - 
Rozdział 6 Podsumowanie i perspektywy- rozdział jest podsumowaniem pracy. Zaprezentowano w nim wyniki, porównania i wnioski.
W dodatku A coś tam coś tam.

\chapter{Rys historyczny i podstawy teoretyczne}
\label{cha:rysipodstawy}

% potrzeba jakiegos wprowadzenia?

%---------------------------------------------------------------------------

\section{Rys historyczny}
\label{sec:ryshistoryczny}
\subsection{Teoria}
Rozpoznawianie obrazów jest nierozerwalnie związane z badaniami nad sztuczną inteligencją. Jest to dziedzina nauki zajmująca się wykorzystaniem maszyn, przede wszystkim komputerów, do wykonywania zadań wymagających ludzkiej inteligencji. Zastosowania tejże dziedziny są nieustannie poszerzane w związku z dynamicznym rozwojem informatyki. Wiele zagadnień dotychczasowo pojmowanych jako problemu wymagające zachowań inteligentnych straciło na znaczeniu na skutek rozwoju technologii. Zmiany te nastąpiły w skutek nieprecyzyjnej definicji inteligencji. Wiele problemów klasyfikowanych jako typowe zagadnienia sztucznej inteligencji obecnie stanowi zagadnienia poruszane w szkołach podczas nauki podstaw algorytmiki. Najlepszym tego przykładem jest zagadnienie "Wież Hanoi". Jednak sztuczna inteligencja nadal zajmuje się problematyką rozpoznawania obrazów.
Polskie tłumaczenie rozpoznawanie obrazów, z angielskiego pattern recognition nie pozwala uchwycić całokształtu tego zagadnienia. Może się ona kojarzyć z przetwarzaniem zdjęć w dwóch wymiarach lub scen wirtualnej rzeczywistości opierającej się na trzech wymiarach. Możliwe, że bardziej trafnym byłoby tu dosłowne tłumaczenie, tj. rozpoznawanie wzorców. Pojęcie rozpoznawania obrazów zostało właściwie sprecyzowane w skrypcie Ryszarda Tadeusiewicza i Mariusza Flasińskiego [??]:

"Ogólnie w zadaniu rozpoznawania obrazów chodzi o rozpoznawanie przynależności rozmaitego
typu obiektów (lub zjawisk) do pewnych klas. Rozpoznawanie to ma być prowadzone
w sytuacji braku apriorycznej informacji na temat reguł przynależności obiektów do
poszczególnych klas, a jedyna informacja możliwa do wykorzystania przez algorytm lub
maszynę rozpoznającą jest zawarta w ciągu uczącym, złożonym z obiektów, dla których
znana jest prawidłowa klasyfikacja."

Maszyna odpowiedzialna za rozpoznawanie obrazów ma za zadanie klasyfikować obiekty biorąc pod uwagę wiedzę o kilku reprezentantach - zbiorze uczącym. Człowiek nie jest w stanie do końca określić sposobu w jaki sam rozpoznaje przedmioty czy kształty na obrazach, zatem nie jest też w stanie przekazać maszynie dokładnych algorytmów działania. Oznacza to, że dziedzina rozpoznawania obrazów jest ciągle poligonem doświadczalnym, na którym ludzie starają się przekazać maszynie jak najlepszy sposób rozpoznawania obiektów. Niestety powoduje to, że zastosowanie rozwiązań teoretycznie sprawdzających się doskonale jest mocno ograniczone.

Niezależnie od wybranej metody rozpoznawanie obrazów rozpoczynamy zawsze od tych samych podstaw. W przedstawionym przypadku zastosowano podejście strukturalne, tj. wybrano wyróżniające obiekt cechy (wyróżniający się kolor oraz kształt) i wykorzystano je przy poszukiwaniu obiektu na obrazie.

\subsection{Rozpoznawanie obrazów w detekcji znaków}

Może będzie tu historia researchu rozpoznawiania znaków ale zależy ile będzie stron. suspiszus

\subsection{Podstawy przetwarzania obrazów}
Kluczową cechą komputerowych systemów przetwarzania obrazów jest dyskretyzacja rzeczywistości. Spowodowane jest to ograniczoną rozdzieczością przetworników analogowo-cyfrowych, takich jak aparaty, kamery czy mierniki laserowe. Obrazy pobrane przy pomocy m.in. powyższych urządzeń są reprezentowane przy pomocy siatki kwadratowej lub (rzadziej) siatki heksagonalnej o skończonym rozmiarze. Siatka heksagonalna pozwala na reprezentację odpowiadającą obrazom przetwarzanym przez ludzki narząd wzroku(rozmieszczenie receptorów w gałce ocznej człowieka można najwłaściwiej odwzorować siatką heksagonalną). O ile jest to bardziej "naturalne" rozwiązanie, to niestety jest ono mniej popularne. Podyktowane jest to przede wszystkim bardziej złożonymi algorytmami, przetwarzającymi obrazy heksagonalne, czy też mniejszą intuicyjnością takich implementacji.
Alternatywą dla siatki heksagonalnej jest siatka kwadratowa. Jest ona o wiele prostsza i wygodniejsza przy implementacji algorytmów przetwarzania obrazów.
Mówiąc o dyskretyzacji rzeczywistości należy wprowadzić pojęcie rozdzielczości obrazu. Wynika ona bezpośrednio z rozdzielczości urządzeń pobierających obraz, zatem nie może być nieograniczona. Definicję rozdzielczości możemy przytoczyć z pracy Tadeusiewicza i Korohody[33]:
Wyraża się ona ilością elementów podstawowych składających się na obraz. Najczęściej
przy płaskich obrazach o kwadratowej siatce zapisywana jest ona jako iloczyn ilości elementów
w poziomie i pionie obrazu.
Dobranie optymalnej rozdzielczości pozwala na zachowanie szczegółów obrazu kluczowych do ich przetwarzania przy, a za razem umożliwia dostosowania wielkości danych wejściowych(obrazów) pozwalając na kontrolę czasu przetwarzania obrazów. Z jednej strony chcielibyśmy mieć jak najlepsze dane pozwalające łatwe na wyekstrahowanie cech obiektów na obrazie, lecz z drugiej strony algorytmy przetwarzania obrazów są zwykle bardzo czasochłonne, co zmusza do ograniczenia rozdzielczości, aby algorytmy wykonywały się w sensownym czasie. Konstruktor systemu przetwarzania obrazów niejednokrotnie musi iść na kompromis w tej kwestii.
Każda najmniejsza część obrazu cyfrowego(piksel) przechowuje swoją wartość, która oznacza jej kolor. Kolor może być jedną z ustalonych wartości. O dziedzinie tej wartość wybrana decyduje przestrzeń kolorów w jakiej obraz został zapisany i łączy się to z ilością bitów jakie przeznaczymy na zapis koloru pojedynczego piksela. Tu też niejednokrotnie trzeba pójść na kompromis. Najpopularniejsze rodzaje obrazów to:
\textbf{binarny} - informacja o kolorze zapisana na 1 bicie. Zapis najprostszy i zajmujący najmniej pamięci. Ułatwia tworzenie algorytmów operujących na takim obrazie i dramatycznie przyspiesza obliczenia.
\textbf{monochromatyczny} - informacja o kolorze zapisana jest na 8 bitach. Przechowują one informację o kolorze z dziedziny 256 odcieni szarości.
\textbf{kolorowy} - informacja o kolorze zapisywana jest na 24 bitach. Pozwala to na przechowywanie informacji o jednym z 16,7 milionów kolorów.

%---------------------------------------------------------------------------


\chapter{Badania literaturowe}

Ten rozdział jest niezbedny we wszystkich pracach dyplomowych i w wiekszosci projektów. W
przypadku, gdy projekt polega na rozbudowie istniejacego algorytmu własnie tutaj nalezy
krótko opisac algorytm zródłowy. W pracach dyplomowych w tym rozdziale nalezy zamiescic
informacje w jaki sposób podchodzi sie do rozwiazywanego problemu na swiecie. Tu mozna
takze zamiescic uwagi krytyczne wobec omawianych prac lub algorytmów, które sa wstepnym
uzasadnieniem dla wprowadzanych pózniej zmian lub dla koniecznosci tworzenia własnego
algorytmu.

Brak sensownej zawartości ;)



\chapter{Koncepcja proponowanego rozwiązania}

Ta czesc pracy przeznaczona jest na prezentacja ogólnej koncepcji rozwiazania problemu,
wraz z jej uzasadnieniem (dlaczego wybrano takie a nie inne metody). Jezli zachodzi taka
potrzeba rozdział nalezy rozbic na podrozdziały - tak, aby opisac po kolei poszczególne fazy
algorytmu. Algorytm nalezy opisac i uzasadnic słownie (ewentualnie wspierajac sie wzorami) a
nastepnie zilustrowac w formie przedstawienia kolejnych czynnosci do realizacji, zapisu w
pseudo kodzie lub w formie schematu blokowego.

Brak sensownej zawartości ;)

Jako rozwiązanie zaimplementowano dwa algorytmy których celem będzie wyodrębnienie poszukiwanych znaków kolejowych W11p z zadanych obrazów. Wyniki tych algorytów zostaną przemielone przez dopasowanie do wzorca. Obrazy testowe oraz maski zostały dostarczone przez promotora. 

Jako pierwszy zaimplenemtowany został alogrytm progowania obrazu wykorzystujący przestrzeń HSV oraz progi dla każdego z kanałów otrzymane metodą prób i błędów.

Drugim z algorytmów jest algorytm centroidów (k-srednich). Tu powinien znaleźć się opis algorytmu.s
\chapter{Realizacja i wyniki testowania}

Ten rozdział powinien zawierac informacje techniczne i wykonawcze, dotyczace tworzonego
rozwiazania, takie jak:
·  uwarunkowania sprzetowe dla rozwiazywanego problemu,
·  wybór systemu operacyjnego i jezyka programowania,
·  dokładny opis działania poszczególnych czesci algorytmu (wzory, zastosowane maski,
współczynniki),
·  sposób generacji lub akwizycji danych do testowania,
·  ogólny opis oprogramowania i warunków, w jakich było ono testowane,
·  wyniki działania programu ze szczególnym uwzglednieniem analizy błedów i ich przyczyn.


Brak sensownej zawartości ;)



\chapter{Podsumowanie i perspektywy}

W tym punkcie nalezy skrótowo przedstawic cała prace (w przypadku projektu dosłownie
kilka zdan) oraz osiagniete wyniki. Na zakonczenie konieczne jest krytyczne przyjrzenie sie
swojej pracy i zaproponowanie kierunków kolejnych modyfikacji lub wskazanie innych metod,
mozliwych do zastosowania w przyszłosci.


Brak sensownej zawartości ;)



\chapter{Instrukcja użytkowania oprogramowania}

Powinna to byl krótka i zwarta instrukcja uzytkowania programu, napisana dla przyszłego uzytkownika, o którym zakłada sie, ze ma dosc blade pojecie o szczegółach algorytmu, a który powinien umiec uruchomic i prawidłowo uzyc stworzony w ramach projektu program. W szczególnosci powinien byc wyjasniony cel, do którego sie dazy oraz sporzadzona lista
wstepnych warunków koniecznych, które musza byc spełnione (system operacyjny, instalacja, rozdzielczosc grafiki, pliki wejsciowe itp.). Nastepnie krok po kroku powinno byc objasnione uzytkowanie programu, w krytycznych miejscach zilustrowane zrzutami okien ekranu. Najlepiej jest to zrobic najpierw ogólnie, a pózniej na wybranym, konkretnym przykładzie.


Brak sensownej zawartości ;)



\chapter{Opis informatyczny procedur}

Punkt ten ma charakter scisle techniczny. Powinny go rozpoczac informacje o srodowisku
programowania, ew. modularyzacji i opcjach kompilacji, plikach, które musza byc dołaczone
oraz uzytych „obcych” bibliotekach. Nastapnie nalezy zamiescic opisy głównych procedur
według standardu zamieszczonego nizej, oraz wymienic niezbedne do ich prawidłowego
działania procedury pomocnicze.
Uwaga: krytyczne fragmenty kodów zródłowych musza byc zaopatrzone w komentarz
\chapter{Spis zawartoci dolaczonych nosnikow}

W poszczególnych katalogach nosnika nalezy umiescic:
·  w zaleznosci od rodzaju projektu - przykładowe lub wszystkie dane (obrazy, filmy), bedace podstawa tworzenia i testowania algorytmu (z ew. odwołaniem sie do materiału analogowego - tytułu nagranej kasety video),
·  SRC - postacie zródłowe stworzonych procedur wraz z projektem, makefile’m itp.
·  EXE - postac programu gotowa do uruchomienia wraz z ew. plikami konfiguracyjnymi lub
innymi niezbednymi komponentami,
·  DOC - tekst raportu w postaci elektronicznej edytowalnej oraz jako PDF (pdf jest
wymagany przez dziekanat).
·  LITERATURA – materiały w postaci elektronicznej (artykuły, inne projekty) które zostały
zgromadzone w zwiazku z realizowanym projektem (praca dyplomowa).

% itd.
% \appendix
% \include{dodatekA}
% \include{dodatekB}
% itd.

\bibliographystyle{alpha}
\bibliography{Literatura}
%\begin{thebibliography}{1}
%
%\bibitem{Dil00}
%A.~Diller.
%\newblock {\em LaTeX wiersz po wierszu}.
%\newblock Wydawnictwo Helion, Gliwice, 2000.
%
%\bibitem{Lam92}
%L.~Lamport.
%\newblock {\em LaTeX system przygotowywania dokumentów}.
%\newblock Wydawnictwo Ariel, Krakow, 1992.
%
%\bibitem{Alvis2011}
%M.~Szpyrka.
%\newblock {\em {On Line Alvis Manual}}.
%\newblock AGH University of Science and Technology, 2011.cccccc
%\newblock \\\texttt{http://fm.ia.agh.edu.pl/alvis:manual}.
%
%\end{thebibliography}

\end{document}
